%%%%%%%%%%%%%%%%%%%%%%%%%%%%%%%%%%%%%%%%%
% Wenneker Article
% LaTeX Template
% Version 2.0 (28/2/17)
%
% This template was downloaded from:
% http://www.LaTeXTemplates.com
%
% Authors:
% Vel (vel@LaTeXTemplates.com)
% Frits Wenneker
%
% License:
% CC BY-NC-SA 3.0 (http://creativecommons.org/licenses/by-nc-sa/3.0/)
%
%%%%%%%%%%%%%%%%%%%%%%%%%%%%%%%%%%%%%%%%%

%----------------------------------------------------------------------------------------
%	PACKAGES AND OTHER DOCUMENT CONFIGURATIONS
%----------------------------------------------------------------------------------------

\documentclass[10pt, a4paper]{ctexart} % 10pt font size (11 and 12 also possible), A4 paper (letterpaper for US letter) and two column layout (remove for one column)

\input{structure.tex} % Specifies the document structure and loads requires packages

%----------------------------------------------------------------------------------------
%	ARTICLE INFORMATION
%----------------------------------------------------------------------------------------

\title{门票倒卖} % The article title

\author{
	\authorstyle{鲁晓东\textsuperscript{1}} % Authors
	\newline\newline % Space before institutions
	\textsuperscript{1}\institution{中山大学岭南学院经济学系}\\ % Institution 1
	%\textsuperscript{2}\institution{University of Texas at Austin, Texas, United States of America}\\ % Institution 2
	%\textsuperscript{3}\institution{\texttt{LaTeXTemplates.com}} % Institution 3
}

% Example of a one line author/institution relationship
%\author{\newauthor{John Marston} \newinstitution{Universidad Nacional Autónoma de México, Mexico City, Mexico}}

\date{\today} % Add a date here if you would like one to appear underneath the title block, use \today for the current date, leave empty for no date
%------------加水印-------------------------------------------
\usepackage{draftwatermark}         % 所有页加水印
%\usepackage[firstpage]{draftwatermark} % 只有第一页加水印
%\SetWatermarkText{Xiaodong Lu}           % 设置水印内容
\SetWatermarkText{\includegraphics{fig//lingnan.png}}         % 设置水印logo
\SetWatermarkLightness{0.9}             % 设置水印透明度 0-1
\SetWatermarkScale{0.5}                   % 设置水印大小 0-1    
%--------------------------------------------------------------

%----------------------------------------------------------------------------------------

\begin{document}

\maketitle % Print the title

\thispagestyle{firstpage} % Apply the page style for the first page (no headers and footers)

%----------------------------------------------------------------------------------------
%	ABSTRACT
%----------------------------------------------------------------------------------------

	
	\lettrineabstract{大型音乐会或者体育赛事的门票往往不是拍卖给出价最高的人。售票人往往把大部分票以“合理”的价格卖出去,再以先到先得或限制每人可买的张数的原则,向由此导致的超额需求来配给剩余的票。这种供票方式为二手市场上以高价卖票创造了可能性这就是门票“倒卖”。}
	




%----------------------------------------------------------------------------------------
%	ARTICLE CONTENTS
%----------------------------------------------------------------------------------------



\section*{图形分析}



图\ref{ticket}画出了倒票的动机,比如美国橄榄球超级碗(bow)大赛的门票。给定消费者的收入和门票报价,他本来想买四张票($A$点)。但全国橄榄球协会限制每人只能买一张票。这一限制使消费者的效用从U2(如果可以自由购票所享有的效用)降到$U_1$。注意到选择一张票($B$点)不符合效用最大化的相切原则在给定票价的情况下,消费者想买的票不止一张。事实上,这个沮丧的消费者现在愿意以高于普通票价的代价去再买几张超级碗门票。他(或她)不仅愿意以官方价格再买一张票(因为$C$点在$U_1$上方),而且愿意放弃一定量其他物品(取决于线段$CD$的长度)去买这张票。看起来此人是非常乐意向“黄牛”出大笔钱了。比如,1996年亚特兰大奥运会主要赛事的门票经常以五倍于面值的价格售出,2005年超级碗上爱国者队的铁杆球迷愿意以多于2000美元的价格去买二手门票

\begin{figure}[ht]
	\centering
	\includegraphics[scale=0.6]{fig//ticket.png} % Figure image
	\caption{门票配给导致黄牛\\考虑到消费者的收入和票价,他(或她)偏好购买4张票。当只能购买到1张票时,效用水平跌落至$U_1$。此人愿意放弃线段$CD$长度的其他物品以原价换取第二张票。。} % Figure caption
	\label{ticket} % Label for referencing with \ref{bear}
\end{figure}

\section*{反倒卖法}
大多数经济学家对倒票抱以相对宽容的态度,认为这是买卖双方的自愿行为。国家和地方政府一般不这么看。很多地方立法管制倒票价格,或严禁在场馆附近倒票。这类法律一般以倒卖“有失公平”作为理由—也许认为票贩子赚取了不应得的利润。然而这种价值判断看起来过于苛刻。票贩子的确提供了某种有价值的服务,使得对门票价值看法不一的人进行交易成为可能。这种交易也方便了临时情况发生变化的人们。禁止这类交易可能导致资源浪费,因为有些座位可能会空着。反倒卖法的主要获益者可能是代理售票点,它们可以获得垄断地位,成为人们获得门票的唯一途径



%---------------------------------问题
\section*{问题}
\begin{enumerate}
	\item 一些体育比赛的举办经营者支持反倒卖法,因为他们认为倒票会侵蚀他们的利润,你怎么看这个论断?
	\item 当物品以被价格以外的手段进行配给时,“黑市”就会兴起。门票倒卖就是一个例子。还有什么别的例子吗?	
	\item 黑市交易是不是一定不好呢?
	
\end{enumerate}

%----------------------------------------------------------------------------------------
%	BIBLIOGRAPHY
%----------------------------------------------------------------------------------------

%\printbibliography[title={Bibliography}] % Print the bibliography, section title in curly brackets

%----------------------------------------------------------------------------------------

%%%%%%%%%%%%%%%%%%%%%%%%%%%%%%%%%%%%%%%%%%%%%%%%%%%%%%%%%%%%%%%%
% 参考文献
%%%%%%%%%%%%%%%%%%%%%%%%%%%%%%%%%%%%%%%%%%%%%%%%%%%%%%%%%%%%%%%%
\small
%99为最大引用条数
\begin{thebibliography}{99}
	\setlength{\parskip}{0pt} %段落之间的竖直距离
	%\bibitem{ref1}  Alex Hiam:《企业总裁:高层经理决策方法)( Englewood Cliffs,N: Prentice Hal,199,20-272页
	\bibitem{ref2} Nicholson, W. and Snyder, C. (2010). Intermediate Microeconomics and Its Application (11 ed.): Cengage Learning. Chap.2
	\end {thebibliography}

\end{document}
