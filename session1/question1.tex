%%%%%%%%%%%%%%%%%%%%%%%%%%%%%%%%%%%%%%%%%
% Wenneker Article
% LaTeX Template
% Version 2.0 (28/2/17)
%
% This template was downloaded from:
% http://www.LaTeXTemplates.com
%
% Authors:
% Vel (vel@LaTeXTemplates.com)
% Frits Wenneker
%
% License:
% CC BY-NC-SA 3.0 (http://creativecommons.org/licenses/by-nc-sa/3.0/)
%
%%%%%%%%%%%%%%%%%%%%%%%%%%%%%%%%%%%%%%%%%

%----------------------------------------------------------------------------------------
%	PACKAGES AND OTHER DOCUMENT CONFIGURATIONS
%----------------------------------------------------------------------------------------

\documentclass[10pt, a4paper]{ctexart} % 10pt font size (11 and 12 also possible), A4 paper (letterpaper for US letter) and two column layout (remove for one column)

\input{structure.tex} % Specifies the document structure and loads requires packages

%----------------------------------------------------------------------------------------
%	ARTICLE INFORMATION
%----------------------------------------------------------------------------------------

\title{练习:看演唱会的机会成本}%The article title

\author{
	\authorstyle{鲁晓东\textsuperscript{1}} % Authors
	\newline\newline % Space before institutions
	\textsuperscript{1}\institution{中山大学岭南学院经济学系}\\ % Institution 1
	%\textsuperscript{2}\institution{University of Texas at Austin, Texas, United States of America}\\ % Institution 2
	%\textsuperscript{3}\institution{\texttt{LaTeXTemplates.com}} % Institution 3
}

% Example of a one line author/institution relationship
%\author{\newauthor{John Marston} \newinstitution{Universidad Nacional Autónoma de México, Mexico City, Mexico}}

\date{\today} % Add a date here if you would like one to appear underneath the title block, use \today for the current date, leave empty for no date
%------------加水印-------------------------------------------
\usepackage{draftwatermark}         % 所有页加水印
%\usepackage[firstpage]{draftwatermark} % 只有第一页加水印
%\SetWatermarkText{Xiaodong Lu}           % 设置水印内容
\SetWatermarkText{\includegraphics{fig//lingnan.png}}         % 设置水印logo
\SetWatermarkLightness{0.9}             % 设置水印透明度 0-1
\SetWatermarkScale{0.5}                   % 设置水印大小 0-1    
%--------------------------------------------------------------

%----------------------------------------------------------------------------------------

\begin{document}

\maketitle % Print the title

\thispagestyle{firstpage} % Apply the page style for the first page (no headers and footers)

%----------------------------------------------------------------------------------------
%	ABSTRACT
%----------------------------------------------------------------------------------------

	
	\lettrineabstract{这是一个“看似简单,其实也不难”的题目,但是,很多人却在“阴沟里翻了船”,甚至包含许多以此为生的职业经济学家。这是一道什么样的题目呢?我们一起来看一下。}
	




%----------------------------------------------------------------------------------------
%	ARTICLE CONTENTS
%----------------------------------------------------------------------------------------



\section*{问题}



假设你赢了一张美国大歌星埃里克·克莱普顿(Eric Clapton)今晚演唱会的免费门票。注意,你不能转售。可另一美国大歌星鲍勃·迪伦(Bob Dylan)今晚也在开演唱会,你也很想去。迪伦的演唱会票价为40美元。当然,你别的时候去看他的演出也行,但你的心理承受价格是50美元。换言之,要是迪伦的票价高过50美元,你就情愿不看了,哪怕你没别的事要做。除此之外,看两人的演出并无其他成本。

试问,你去看克莱普顿演唱会的机会成本是多少? 

\begin{enumerate}[A.]
	\item 0美元
	\item 10美元
	\item 40美元
	\item 50美元
\end{enumerate}


\section*{花絮}
这个问题最早由美国经济学家费雷罗和泰勒设计。最初,他们向270名最近上了经济学课程的大学生提出了这个问题,只有7.4\%的人选择了正确答案。因为只有4个选项,哪怕学生们是随机选择,正确率也该有25\%。看起来,同学们似乎觉得掌握这些知识很丢脸。

之后,费雷罗和泰勒又向88名从没上过经济学课程的学生提出了同一个问题,这回的正确率是17.2\%--比上过经济学课程的学生高两倍,但仍比随机选择的正确率要低。 

为什么上过经济学课程的学生没能表现更佳呢?窃以为,主要原因是,在典型的概论课上,教授会给学生灌输几百个概念,机会成本只是其中之一,而且模模糊糊,一笔带过。倘若学生没花足够的时间在上面,没在不同的例子里反复演练,也就无法真正理解它。 

但费雷罗和泰勒提出了另一种可能性:教经济学的讲师自己也没掌握机会成本的基本概念。2005年美国经济学协会\footnote{最顶级的经济学家集会,相当于经济学界的华山论剑,许多受邀经济学家一生以此为荣。}开年会的时候,他俩向199名专业经济学家提出了同一个问题,只有21.6\%的人选择了正确答案,25.1\%的人认为去看克莱普顿演唱会的机会成本是0,25.6\%认为是40美元,还有 27.6\%认为是50美元。 

费雷罗和泰勒核查了经济学概论最主要的教科书,他们发现,大多数教科书对机会成本并未给予足够的重视,来帮助学生解答克莱普顿/迪伦问题。他们还注意到,比概论程度更深的教科书也未曾耐心、深入地介绍这个概念。在大学微观经济学的首选课本当中,索引里都找不到"机会成本"这个词。 

机会成本的概念在经济学发挥了“基石”的作用,它直接来源于经济学的“资源稀缺”的benchmark假设,无数的经济学原理就是建立在“机会成本”之上的,这其中就包括“比较优势”的概念。因此,如果不能正确理解“机会成本”的概念,那么亚当斯密的劳动分工的思想将不在成立,而基于分工的全球贸易体系也将不复存在。另外,比较优势的概念对于个人的来说也意义非凡。给每一个人以存在的确定性理由,无论这个人是聪明还是愚蠢,富裕还是贫穷,博学还是无知。可以说,正确的理解机会成本之后,整个周遭的世界以及人生会瞬间明亮起来。这是你的老师真实的感受,不信你试试。




\end{document}
